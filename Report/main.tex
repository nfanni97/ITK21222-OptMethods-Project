\documentclass{article}

\author{Fanni Dorottya Nagy\\F197A4}
\date{May 8, 2022}
\title{Optimization Methods Assignment Documentation}

\begin{document}
\maketitle
\section{Problem}
First task, complex transportation problem.

A company produces computers in 8 cities, A1,…,A8.
The amount that they can produce per year (at most) is as follows: 280, 400, 3000, 240, 1200, 430,
2300, 650.
The cost of production is as follows (per unit): 250, 650, 340, 970, 650, 340, 450, 650.
They transport the computers to 9 cities, C1,…,C9.
The demand is as follows: 230, 130, 250, 345, 650, 350, 320, 540, 1200.

There is also some initial cost if production is made, it is 2000 dollar at any factory.
The transportation goes through the next cities: B1,…,B5.

It also holds that in the intermediate places (B1,…,B5) at most the following units can be
transported: 1300, 1200, 750, 790, 2300.

Let us determine the optimal transportation plan ( how much should be transported from what place
to what place through what cities) so that the total cost is minimum!

Transportation costs are found in the csv files in the attached repository.

\section{Discussion}
\subsection{Selected Algorithm}
The problem is solved by the "Simulated Annealing" algorithm as the chosen neighborhood structure was best suited for randomly generated neighbors instead of an exhaustive search demanded by Tabu Search.
The meta parameters are the number of iterations, the initial temperature and the value with which it should be decreased in each iteration.
Stopping criteria is simply that the set number of iterations have been completed.

\subsection{Neighborhood definition}
The simplest neighborhood definition was chosen that I could come up with.
$x_1$ and $x_2$ are neighbors if we can get $x_2$ from $x_1$ by choosing a transport in the first layer (A-\textgreater B) and moving some of its units from another source instead.

\textbf{Example}

Let us assume that we are currently at the following solution:
\begin{itemize}
    \item 280 from A1 to B1
    \item 400 from A2 to B1
    \item 620 from A3 to B1
    \item 1200 from A3 to B2
    \item 750 from A3 to B3
    \item 430 from A3 to B4
    \item 240 from A4 to B4
    \item 95 from A5 to B4
    \item 230 from B1 to C1
    \item 130 from B1 to C2
    \item 250 from B1 to C3
    \item 345 from B1 to C4
    \item 345 from B1 to C5
    \item 305 from B2 to C5
    \item 350 from B2 to C6
    \item 320 from B2 to C7
    \item 225 from B2 to C8
    \item 315 from B3 to C8
    \item 435 from B3 to C9
    \item 765 from B4 to C9
\end{itemize}
Additionally, assume that the random generator selected transport A3-\textgreater B1 and the city A8.
Then, at most $min(620,3000)$ can be moved: 620 was transported from this location, while at most 3000 can be moved to A8.
From this, the amount to be transferred from A8 instead of from A3 is also selected randomly, let's say 598.
Then, the original transport is modified and another one is added:
\begin{itemize}
    \item 22 from A3 to B1
    \item 598 from A8 to B1
\end{itemize}

\subsection{Result}
As there are multiple randomized steps in the neighborhood generator (selecting transport, selecting new city, amount to be moved) the solution will always be different.
Hence it is not possible to know if the solution after the last iteration is optimal.

In the provided code, the algorithm is run 15 times with the same hyperparameters.
It was found that while the cost of the implemented greedy solution was 260 790, the mean of the 15 runs is around 225 000 and the minimum is around 210-215 000.
\end{document}